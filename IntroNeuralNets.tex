% Options for packages loaded elsewhere
\PassOptionsToPackage{unicode}{hyperref}
\PassOptionsToPackage{hyphens}{url}
%
\documentclass[
]{article}
\usepackage{lmodern}
\usepackage{amssymb,amsmath}
\usepackage{ifxetex,ifluatex}
\ifnum 0\ifxetex 1\fi\ifluatex 1\fi=0 % if pdftex
  \usepackage[T1]{fontenc}
  \usepackage[utf8]{inputenc}
  \usepackage{textcomp} % provide euro and other symbols
\else % if luatex or xetex
  \usepackage{unicode-math}
  \defaultfontfeatures{Scale=MatchLowercase}
  \defaultfontfeatures[\rmfamily]{Ligatures=TeX,Scale=1}
\fi
% Use upquote if available, for straight quotes in verbatim environments
\IfFileExists{upquote.sty}{\usepackage{upquote}}{}
\IfFileExists{microtype.sty}{% use microtype if available
  \usepackage[]{microtype}
  \UseMicrotypeSet[protrusion]{basicmath} % disable protrusion for tt fonts
}{}
\makeatletter
\@ifundefined{KOMAClassName}{% if non-KOMA class
  \IfFileExists{parskip.sty}{%
    \usepackage{parskip}
  }{% else
    \setlength{\parindent}{0pt}
    \setlength{\parskip}{6pt plus 2pt minus 1pt}}
}{% if KOMA class
  \KOMAoptions{parskip=half}}
\makeatother
\usepackage{xcolor}
\IfFileExists{xurl.sty}{\usepackage{xurl}}{} % add URL line breaks if available
\IfFileExists{bookmark.sty}{\usepackage{bookmark}}{\usepackage{hyperref}}
\hypersetup{
  pdftitle={NeuralNetwork},
  hidelinks,
  pdfcreator={LaTeX via pandoc}}
\urlstyle{same} % disable monospaced font for URLs
\usepackage[margin=1in]{geometry}
\usepackage{color}
\usepackage{fancyvrb}
\newcommand{\VerbBar}{|}
\newcommand{\VERB}{\Verb[commandchars=\\\{\}]}
\DefineVerbatimEnvironment{Highlighting}{Verbatim}{commandchars=\\\{\}}
% Add ',fontsize=\small' for more characters per line
\usepackage{framed}
\definecolor{shadecolor}{RGB}{248,248,248}
\newenvironment{Shaded}{\begin{snugshade}}{\end{snugshade}}
\newcommand{\AlertTok}[1]{\textcolor[rgb]{0.94,0.16,0.16}{#1}}
\newcommand{\AnnotationTok}[1]{\textcolor[rgb]{0.56,0.35,0.01}{\textbf{\textit{#1}}}}
\newcommand{\AttributeTok}[1]{\textcolor[rgb]{0.77,0.63,0.00}{#1}}
\newcommand{\BaseNTok}[1]{\textcolor[rgb]{0.00,0.00,0.81}{#1}}
\newcommand{\BuiltInTok}[1]{#1}
\newcommand{\CharTok}[1]{\textcolor[rgb]{0.31,0.60,0.02}{#1}}
\newcommand{\CommentTok}[1]{\textcolor[rgb]{0.56,0.35,0.01}{\textit{#1}}}
\newcommand{\CommentVarTok}[1]{\textcolor[rgb]{0.56,0.35,0.01}{\textbf{\textit{#1}}}}
\newcommand{\ConstantTok}[1]{\textcolor[rgb]{0.00,0.00,0.00}{#1}}
\newcommand{\ControlFlowTok}[1]{\textcolor[rgb]{0.13,0.29,0.53}{\textbf{#1}}}
\newcommand{\DataTypeTok}[1]{\textcolor[rgb]{0.13,0.29,0.53}{#1}}
\newcommand{\DecValTok}[1]{\textcolor[rgb]{0.00,0.00,0.81}{#1}}
\newcommand{\DocumentationTok}[1]{\textcolor[rgb]{0.56,0.35,0.01}{\textbf{\textit{#1}}}}
\newcommand{\ErrorTok}[1]{\textcolor[rgb]{0.64,0.00,0.00}{\textbf{#1}}}
\newcommand{\ExtensionTok}[1]{#1}
\newcommand{\FloatTok}[1]{\textcolor[rgb]{0.00,0.00,0.81}{#1}}
\newcommand{\FunctionTok}[1]{\textcolor[rgb]{0.00,0.00,0.00}{#1}}
\newcommand{\ImportTok}[1]{#1}
\newcommand{\InformationTok}[1]{\textcolor[rgb]{0.56,0.35,0.01}{\textbf{\textit{#1}}}}
\newcommand{\KeywordTok}[1]{\textcolor[rgb]{0.13,0.29,0.53}{\textbf{#1}}}
\newcommand{\NormalTok}[1]{#1}
\newcommand{\OperatorTok}[1]{\textcolor[rgb]{0.81,0.36,0.00}{\textbf{#1}}}
\newcommand{\OtherTok}[1]{\textcolor[rgb]{0.56,0.35,0.01}{#1}}
\newcommand{\PreprocessorTok}[1]{\textcolor[rgb]{0.56,0.35,0.01}{\textit{#1}}}
\newcommand{\RegionMarkerTok}[1]{#1}
\newcommand{\SpecialCharTok}[1]{\textcolor[rgb]{0.00,0.00,0.00}{#1}}
\newcommand{\SpecialStringTok}[1]{\textcolor[rgb]{0.31,0.60,0.02}{#1}}
\newcommand{\StringTok}[1]{\textcolor[rgb]{0.31,0.60,0.02}{#1}}
\newcommand{\VariableTok}[1]{\textcolor[rgb]{0.00,0.00,0.00}{#1}}
\newcommand{\VerbatimStringTok}[1]{\textcolor[rgb]{0.31,0.60,0.02}{#1}}
\newcommand{\WarningTok}[1]{\textcolor[rgb]{0.56,0.35,0.01}{\textbf{\textit{#1}}}}
\usepackage{graphicx,grffile}
\makeatletter
\def\maxwidth{\ifdim\Gin@nat@width>\linewidth\linewidth\else\Gin@nat@width\fi}
\def\maxheight{\ifdim\Gin@nat@height>\textheight\textheight\else\Gin@nat@height\fi}
\makeatother
% Scale images if necessary, so that they will not overflow the page
% margins by default, and it is still possible to overwrite the defaults
% using explicit options in \includegraphics[width, height, ...]{}
\setkeys{Gin}{width=\maxwidth,height=\maxheight,keepaspectratio}
% Set default figure placement to htbp
\makeatletter
\def\fps@figure{htbp}
\makeatother
\setlength{\emergencystretch}{3em} % prevent overfull lines
\providecommand{\tightlist}{%
  \setlength{\itemsep}{0pt}\setlength{\parskip}{0pt}}
\setcounter{secnumdepth}{-\maxdimen} % remove section numbering

\title{NeuralNetwork}
\author{}
\date{\vspace{-2.5em}}

\begin{document}
\maketitle

\begin{Shaded}
\begin{Highlighting}[]
\CommentTok{#install.packages("neuralnet")}
\end{Highlighting}
\end{Shaded}

\hypertarget{create-a-dataset-technical-knowledge-score-tkk-and-communication-skills-score-css-as-features-and-placed-as-a-class-label}{%
\section{Create a dataset, Technical Knowledge Score (TKK) and
Communication Skills Score (CSS) as features, and ``placed'' as a class
label}\label{create-a-dataset-technical-knowledge-score-tkk-and-communication-skills-score-css-as-features-and-placed-as-a-class-label}}

\begin{Shaded}
\begin{Highlighting}[]
\CommentTok{# creating training data set}
\NormalTok{TKS=}\KeywordTok{c}\NormalTok{(}\DecValTok{20}\NormalTok{,}\DecValTok{10}\NormalTok{,}\DecValTok{30}\NormalTok{,}\DecValTok{20}\NormalTok{,}\DecValTok{80}\NormalTok{,}\DecValTok{30}\NormalTok{)}
\NormalTok{CSS=}\KeywordTok{c}\NormalTok{(}\DecValTok{90}\NormalTok{,}\DecValTok{20}\NormalTok{,}\DecValTok{40}\NormalTok{,}\DecValTok{50}\NormalTok{,}\DecValTok{50}\NormalTok{,}\DecValTok{80}\NormalTok{)}
\NormalTok{Placed=}\KeywordTok{c}\NormalTok{(}\DecValTok{1}\NormalTok{,}\DecValTok{0}\NormalTok{,}\DecValTok{0}\NormalTok{,}\DecValTok{0}\NormalTok{,}\DecValTok{1}\NormalTok{,}\DecValTok{1}\NormalTok{)}

\CommentTok{# Here, you will combine multiple columns or features into a single set of data}
\NormalTok{df=}\KeywordTok{data.frame}\NormalTok{(TKS,CSS,Placed)}
\end{Highlighting}
\end{Shaded}

\hypertarget{lets-build-a-nn-classifier-model-using-the-neuralnet-library.}{%
\section{Let's build a NN classifier model using the neuralnet
library.}\label{lets-build-a-nn-classifier-model-using-the-neuralnet-library.}}

\hypertarget{first-import-the-neuralnet-library-and-create-nn-classifier-model-by-passing-argument-set-of-label-and-features-dataset-number-of-neurons-in-hidden-layers-and-error-calculation.}{%
\section{First, import the neuralnet library and create NN classifier
model by passing argument set of label and features, dataset, number of
neurons in hidden layers, and error
calculation.}\label{first-import-the-neuralnet-library-and-create-nn-classifier-model-by-passing-argument-set-of-label-and-features-dataset-number-of-neurons-in-hidden-layers-and-error-calculation.}}

\hypertarget{placedtkscss-placed-is-label-annd-tks-and-css-are-features.}{%
\section{- Placed\textasciitilde TKS+CSS, Placed is label annd TKS and
CSS are
features.}\label{placedtkscss-placed-is-label-annd-tks-and-css-are-features.}}

\hypertarget{df-is-dataframe}{%
\section{- df is dataframe,}\label{df-is-dataframe}}

\hypertarget{hidden3-represents-single-layer-with-3-neurons-respectively.}{%
\section{- hidden=3: represents single layer with 3 neurons
respectively.}\label{hidden3-represents-single-layer-with-3-neurons-respectively.}}

\hypertarget{act.fct-logistic-used-for-smoothing-the-result.}{%
\section{- act.fct = ``logistic'' used for smoothing the
result.}\label{act.fct-logistic-used-for-smoothing-the-result.}}

\hypertarget{linear.ouputfalse-set-false-for-apply-act.fct-otherwise-true}{%
\section{- linear.ouput=FALSE: set FALSE for apply act.fct otherwise
TRUE}\label{linear.ouputfalse-set-false-for-apply-act.fct-otherwise-true}}

\begin{Shaded}
\begin{Highlighting}[]
\KeywordTok{require}\NormalTok{(neuralnet)}
\end{Highlighting}
\end{Shaded}

\begin{verbatim}
## Loading required package: neuralnet
\end{verbatim}

\begin{verbatim}
## Warning: package 'neuralnet' was built under R version 3.6.2
\end{verbatim}

\begin{Shaded}
\begin{Highlighting}[]
\CommentTok{# fit neural network}
\NormalTok{nn=}\KeywordTok{neuralnet}\NormalTok{(Placed}\OperatorTok{~}\NormalTok{TKS}\OperatorTok{+}\NormalTok{CSS,}\DataTypeTok{data=}\NormalTok{df, }\DataTypeTok{hidden=}\DecValTok{3}\NormalTok{,}\DataTypeTok{act.fct =} \StringTok{"logistic"}\NormalTok{,}
                \DataTypeTok{linear.output =} \OtherTok{FALSE}\NormalTok{)}
\end{Highlighting}
\end{Shaded}

\hypertarget{plotting-neural-network}{%
\subsection{Plotting Neural Network}\label{plotting-neural-network}}

\hypertarget{lets-plot-your-neural-net-model.including-plots}{%
\section{Let's plot your neural net model.Including
Plots}\label{lets-plot-your-neural-net-model.including-plots}}

\begin{Shaded}
\begin{Highlighting}[]
\CommentTok{# plot neural network}
\KeywordTok{plot}\NormalTok{(nn)}
\end{Highlighting}
\end{Shaded}

\hypertarget{create-test-dataset-using-two-features-technical-knowledge-score-tkk-and-communication-skills-score-css}{%
\section{Create test dataset using two features Technical Knowledge
Score (TKK) and Communication Skills Score
(CSS)}\label{create-test-dataset-using-two-features-technical-knowledge-score-tkk-and-communication-skills-score-css}}

\hypertarget{predict-the-results-for-the-test-set}{%
\section{Predict the results for the test
set}\label{predict-the-results-for-the-test-set}}

\hypertarget{predict-the-probability-score-for-the-test-data-using-the-compute-function.}{%
\section{Predict the probability score for the test data using the
compute
function.}\label{predict-the-probability-score-for-the-test-data-using-the-compute-function.}}

\begin{Shaded}
\begin{Highlighting}[]
\CommentTok{## Prediction using neural network}
\NormalTok{Predict=}\KeywordTok{compute}\NormalTok{(nn,test)}
\NormalTok{Predict}\OperatorTok{$}\NormalTok{net.result}
\end{Highlighting}
\end{Shaded}

\begin{verbatim}
##          [,1]
## [1,] 0.495644
## [2,] 0.495644
## [3,] 0.495644
\end{verbatim}

\hypertarget{converting-probabilities-into-binary-classes-setting-threshold-level-0.5}{%
\section{\# Converting probabilities into binary classes setting
threshold level
0.5}\label{converting-probabilities-into-binary-classes-setting-threshold-level-0.5}}

\begin{Shaded}
\begin{Highlighting}[]
\NormalTok{prob <-}\StringTok{ }\NormalTok{Predict}\OperatorTok{$}\NormalTok{net.result}
\NormalTok{pred <-}\StringTok{ }\KeywordTok{ifelse}\NormalTok{(prob}\OperatorTok{>}\FloatTok{0.5}\NormalTok{, }\DecValTok{1}\NormalTok{, }\DecValTok{0}\NormalTok{)}
\NormalTok{pred}
\end{Highlighting}
\end{Shaded}

\begin{verbatim}
##      [,1]
## [1,]    0
## [2,]    0
## [3,]    0
\end{verbatim}

\#Predicted results are 1,0, and 1.

\# Pros and Cons \#Neural networks are more flexible and can be used
with both regression and classification problems. Neural networks are
\#good for the nonlinear dataset with a large number of inputs such as
images. Neural networks can work with any number of \#inputs and layers.
Neural networks have the numerical strength that can perform jobs in
parallel.

\#There are more alternative algorithms such as SVM, Decision Tree and
Regression are available that are simple, fast, \#easy to train, and
provide better performance. Neural networks are much more of the black
box, require more time for \#development and more computation power.
Neural Networks requires more data than other Machine Learning
algorithms. NNs \#can be used only with numerical inputs and non-missing
value datasets. A well-known neural network researcher said ``A \#neural
network is the second best way to solve any problem. The best way is to
actually understand the problem,''

\# Use-cases of NN

\#NN's wonderful properties offer many applications such as:

\#Pattern Recognition: neural networks are very suitable for pattern
recognition problems such as facial recognition, \#object detection,
fingerprint recognition, etc. \#Anomaly Detection: neural networks are
good at pattern detection, and they can easily detect the unusual
patterns that \#don't fit in the general patterns. \#Time Series
Prediction: Neural networks can be used to predict time series problems
such as stock price, weather \#forecasting. \#Natural Language
Processing: Neural networks offer a wide range of applications in
Natural Language Processing tasks \#such as text classification, Named
Entity Recognition (NER), Part-of-Speech Tagging, Speech Recognition,
and Spell \#Checking.

\#Conclusion \#Congratulations, you have made it to the end of this
tutorial!

\#In this tutorial, you have covered a lot of details about the Neural
Network. You have learned what Neural Network, \#Forward Propagation,
and Back Propagation are, along with Activation Functions,
Implementation of the neural network in \#R, Use-cases of NN, and
finally Pros, and Cons of NN.

\#Hopefully, you can now utilize Neural Network concept to analyze your
own datasets. Thanks for reading this tutorial!

\end{document}
